% the sample slide is created with 16:9 aspect ratio
\documentclass[aspectratio=169]{beamer}

% remove the options if you do not want to have them
\usetheme[
	% background=images/background.jpg, % you can add your own background image
	logo=images/unsw-portrait.png,
	sidelogo=images/unsw-landscape.png,
]{unsw}
% uncomment to show notes. Works very nicely with dspdfviewer. You get something similar to PPT's presenter view.
%\usepackage{pgfpages}
%\setbeameroption{show notes on second screen}

% information for the title page
\author{Luke Wicent Sy}
\title{UNSW Beamer Theme}
\subtitle{An unofficial theme for University of New South Wales}
\institute{Graduate School of Biomedical Engineering}
\date{\today}

\begin{document}
	% use plain option to remove the page number from the title slide
	\begin{frame}[plain]
		\titlepage
	\end{frame}
	
	\begin{frame}{Welcome to Sydney!}
	Sydney, capital of New South Wales and one of Australia's largest cities, is best known for its harbourfront Sydney Opera House, with a distinctive sail-like design.
		\begin{itemize}
			\item Item \textit{item}
				\begin{itemize}
					\item I am a subitem
				\end{itemize}
			\item \textbf{this is a bold text}
			\item \alert{this is an alert}
			\item italics is not supported at the moment
		\end{itemize}
	\end{frame}

	\begin{frame}{Sample Blocks}
		\framesubtitle{This is a subtitle}
		\begin{block}{Standard Block}
			This is a standard block.
		\end{block}
		
		\begin{exampleblock}{Example Block}
			This is an example block.
		\end{exampleblock}
		
		\begin{alertblock}{Alert Block}
			This is an alert block.
		\end{alertblock}
	\end{frame}
	
	\begin{frame}{Math}
		Mathematics is the queen of sciences and arithmetic is the queen of mathematics.

		\begin{align*}
			\vec{x} &= \vec{q}_{s, k} = \vec{R}_{s, k} = 
			\begin{bmatrix}
			\vec{R}_{s, x, k} & \vec{R}_{s, y, k} & \vec{R}_{s, z, k}
			\end{bmatrix} \\
			& \vec{p}_{lhip, k} = \vec{p}_{pelv, k} + d_{pelv}/2*\vec{R}_{pelv, y, k} \\
			& \vec{p}_{lkne, k} = \vec{p}_{lank, k} + d_{ltib}*\vec{R}_{ltib, z, k} \\
			& (\vec{p}_{lhip, k} - \vec{p}_{lkne, k} ) \cdot \vec{R}_{ltib, y, k} = 0\\
			& ||\vec{p}_{lhip, k} - \vec{p}_{lkne, k}||_2 = d_{lfem}
		\end{align*}	
		
		% these notes will only show when you uncomment 
		\note[item]{sample note 1}
		\note[item]{sample note 2}	
		
	\end{frame}

	\begin{frame}{Two Columns}
		We can also add two columns in the slides.
		\begin{columns}[t]
			\begin{column}[T]{0.4\textwidth}
				This is the first column. In this column, we can also add a block for instance.
				\vspace{1em}
				\begin{block}{Block}
					I am a block in a column.
				\end{block}
			\end{column}
			\begin{column}[T]{0.4\textwidth}
				\begin{itemize}
					\item In this column,
					\item we just add the
					\item bullet points.
				\end{itemize}
			\end{column}
		\end{columns}
	\end{frame}
	\begin{frame}{Acknowledgements}
		This theme is based on Kailash Budhathoki's \href{https://github.com/kailashbuki/beamerthemesaarland}{Saarland Beamer Theme}.
		
		Color guide was based from the visual guide found at the \href{https://www.brand.unsw.edu.au/download/}{UNSW brand hub}.
	\end{frame}
\end{document}
